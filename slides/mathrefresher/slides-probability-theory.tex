%<<setup-child, include = FALSE>>=
%library(knitr)
%library(qrmix)
%library(mlr)
%library(quantreg)
%library(reshape2)
%set_parent("../style/preamble.Rnw")
%@

\documentclass[11pt,compress,t,notes=noshow, xcolor=table]{beamer}
\usepackage[]{graphicx}\usepackage[]{color}
% maxwidth is the original width if it is less than linewidth
% otherwise use linewidth (to make sure the graphics do not exceed the margin)
\makeatletter
\def\maxwidth{ %
  \ifdim\Gin@nat@width>\linewidth
    \linewidth
  \else
    \Gin@nat@width
  \fi
}
\makeatother

\definecolor{fgcolor}{rgb}{0.345, 0.345, 0.345}
\newcommand{\hlnum}[1]{\textcolor[rgb]{0.686,0.059,0.569}{#1}}%
\newcommand{\hlstr}[1]{\textcolor[rgb]{0.192,0.494,0.8}{#1}}%
\newcommand{\hlcom}[1]{\textcolor[rgb]{0.678,0.584,0.686}{\textit{#1}}}%
\newcommand{\hlopt}[1]{\textcolor[rgb]{0,0,0}{#1}}%
\newcommand{\hlstd}[1]{\textcolor[rgb]{0.345,0.345,0.345}{#1}}%
\newcommand{\hlkwa}[1]{\textcolor[rgb]{0.161,0.373,0.58}{\textbf{#1}}}%
\newcommand{\hlkwb}[1]{\textcolor[rgb]{0.69,0.353,0.396}{#1}}%
\newcommand{\hlkwc}[1]{\textcolor[rgb]{0.333,0.667,0.333}{#1}}%
\newcommand{\hlkwd}[1]{\textcolor[rgb]{0.737,0.353,0.396}{\textbf{#1}}}%
\let\hlipl\hlkwb

\usepackage{framed}
\makeatletter
\newenvironment{kframe}{%
 \def\at@end@of@kframe{}%
 \ifinner\ifhmode%
  \def\at@end@of@kframe{\end{minipage}}%
  \begin{minipage}{\columnwidth}%
 \fi\fi%
 \def\FrameCommand##1{\hskip\@totalleftmargin \hskip-\fboxsep
 \colorbox{shadecolor}{##1}\hskip-\fboxsep
     % There is no \\@totalrightmargin, so:
     \hskip-\linewidth \hskip-\@totalleftmargin \hskip\columnwidth}%
 \MakeFramed {\advance\hsize-\width
   \@totalleftmargin\z@ \linewidth\hsize
   \@setminipage}}%
 {\par\unskip\endMakeFramed%
 \at@end@of@kframe}
\makeatother

\definecolor{shadecolor}{rgb}{.97, .97, .97}
\definecolor{messagecolor}{rgb}{0, 0, 0}
\definecolor{warningcolor}{rgb}{1, 0, 1}
\definecolor{errorcolor}{rgb}{1, 0, 0}
\newenvironment{knitrout}{}{} % an empty environment to be redefined in TeX

\usepackage{alltt}
\newcommand{\SweaveOpts}[1]{}  % do not interfere with LaTeX
\newcommand{\SweaveInput}[1]{} % because they are not real TeX commands
\newcommand{\Sexpr}[1]{}       % will only be parsed by R
\newcommand{\xmark}{\ding{55}}%


\usepackage[english]{babel}
\usepackage[utf8]{inputenc}

\usepackage{dsfont}
\usepackage{verbatim}
\usepackage{amsmath}
\usepackage{amsfonts}
\usepackage{amssymb}
\usepackage{bm}
\usepackage{csquotes}
\usepackage{multirow}
\usepackage{longtable}
\usepackage{booktabs}
\usepackage{enumerate}
\usepackage[absolute,overlay]{textpos}
\usepackage{psfrag}
\usepackage{algorithm}
\usepackage{algpseudocode}
\usepackage{eqnarray}
\usepackage{arydshln}
\usepackage{tabularx}
\usepackage{placeins}
\usepackage{tikz}
\usepackage{setspace}
\usepackage{colortbl}
\usepackage{mathtools}
\usepackage{wrapfig}
\usepackage{bm}
\usepackage{amsmath}
\usepackage{pifont}
\usepackage[round]{natbib}
\usepackage{hyperref}

\usetikzlibrary{shapes,arrows,automata,positioning,calc,chains,trees, shadows}
\tikzset{
  %Define standard arrow tip
  >=stealth',
  %Define style for boxes
  punkt/.style={
    rectangle,
    rounded corners,
    draw=black, very thick,
    text width=6.5em,
    minimum height=2em,
    text centered},
  % Define arrow style
  pil/.style={
    ->,
    thick,
    shorten <=2pt,
    shorten >=2pt,}
}

\usepackage{subfig}

% Defines macros and environments
\usepackage{../../style/lmu-lecture}


\let\code=\texttt
\let\proglang=\textsf

\setkeys{Gin}{width=0.9\textwidth}

\setbeamertemplate{frametitle}{\expandafter\uppercase\expandafter\insertframetitle}

% basic latex stuff
\newcommand{\pkg}[1]{{\fontseries{b}\selectfont #1}} % fontstyle for R packages

% Often used in exercise Rnw files, still relevant?
\newcommand{\lz}{\vspace{0.5cm}} % vertical space
\newcommand{\dlz}{\vspace{1cm}}  % double vertical space

% Unused and about to be deleted
\newcommand{\oneliner}[1] % Oneliner for important statements
{\begin{block}{}\begin{center}\begin{Large}#1\end{Large}\end{center}\end{block}}


%--------------------%
%  New environments  %
%--------------------%

 % Frame with breaks and verbatim // this is used very often
\newenvironment{vbframe}
{
\begin{frame}[containsverbatim,allowframebreaks]
}
{
\end{frame}
}

% Frame with verbatim without breaks (to avoid numbering one slided frames)
% This is not used anywhere but I can see it being useful
\newenvironment{vframe}
{
\begin{frame}[containsverbatim]
}
{
\end{frame}
}

% Itemize block
\newenvironment{blocki}[1]
{
\begin{block}{#1}\begin{itemize}
}
{
\end{itemize}\end{block}
}

%--------------%
%  Citebutton  %
%--------------%
% Example usage (from slides-cart-discussion.tex)
% \citebutton{Breiman, 1984}{https://www.taylorfrancis.com/books/mono/10.1201/9781315139470/classification-regression-trees-leo-breiman}
\newcommand{\citebutton}[2]{%
\NoCaseChange{\resizebox{!}{9pt}{\protect\beamergotobutton{\href{#2}{#1}}}}%
}

% textcolor that works in mathmode
% https://tex.stackexchange.com/a/261480
% Used e.g. in forests/slides-forests-bagging.tex
% [...] \textcolor{blue}{\tfrac{1}{M}\sum^M_{m} [...]
\makeatletter
\renewcommand*{\@textcolor}[3]{%
  \protect\leavevmode
  \begingroup
    \color#1{#2}#3%
  \endgroup
}
\makeatother





% dependencies: amsmath, amssymb, dsfont
% math spaces
\ifdefined\N
\renewcommand{\N}{\mathds{N}} % N, naturals
\else \newcommand{\N}{\mathds{N}} \fi
\newcommand{\Z}{\mathds{Z}} % Z, integers
\newcommand{\Q}{\mathds{Q}} % Q, rationals
\newcommand{\R}{\mathds{R}} % R, reals
\ifdefined\C
\renewcommand{\C}{\mathds{C}} % C, complex
\else \newcommand{\C}{\mathds{C}} \fi
\newcommand{\continuous}{\mathcal{C}} % C, space of continuous functions
\newcommand{\M}{\mathcal{M}} % machine numbers
\newcommand{\epsm}{\epsilon_m} % maximum error

% counting / finite sets
\newcommand{\setzo}{\{0, 1\}} % set 0, 1
\newcommand{\setmp}{\{-1, +1\}} % set -1, 1
\newcommand{\unitint}{[0, 1]} % unit interval

% basic math stuff
\newcommand{\xt}{\tilde x} % x tilde
\DeclareMathOperator*{\argmax}{arg\,max} % argmax
\DeclareMathOperator*{\argmin}{arg\,min} % argmin
\newcommand{\argminlim}{\mathop{\mathrm{arg\,min}}\limits} % argmax with limits
\newcommand{\argmaxlim}{\mathop{\mathrm{arg\,max}}\limits} % argmin with limits
\newcommand{\sign}{\operatorname{sign}} % sign, signum
\newcommand{\I}{\mathbb{I}} % I, indicator
\newcommand{\order}{\mathcal{O}} % O, order
\newcommand{\bigO}{\mathcal{O}} % Big-O Landau
\newcommand{\littleo}{{o}} % Little-o Landau
\newcommand{\pd}[2]{\frac{\partial{#1}}{\partial #2}} % partial derivative
\newcommand{\floorlr}[1]{\left\lfloor #1 \right\rfloor} % floor
\newcommand{\ceillr}[1]{\left\lceil #1 \right\rceil} % ceiling
\newcommand{\indep}{\perp \!\!\! \perp} % independence symbol

% sums and products
\newcommand{\sumin}{\sum\limits_{i=1}^n} % summation from i=1 to n
\newcommand{\sumim}{\sum\limits_{i=1}^m} % summation from i=1 to m
\newcommand{\sumjn}{\sum\limits_{j=1}^n} % summation from j=1 to p
\newcommand{\sumjp}{\sum\limits_{j=1}^p} % summation from j=1 to p
\newcommand{\sumik}{\sum\limits_{i=1}^k} % summation from i=1 to k
\newcommand{\sumkg}{\sum\limits_{k=1}^g} % summation from k=1 to g
\newcommand{\sumjg}{\sum\limits_{j=1}^g} % summation from j=1 to g
\newcommand{\meanin}{\frac{1}{n} \sum\limits_{i=1}^n} % mean from i=1 to n
\newcommand{\meanim}{\frac{1}{m} \sum\limits_{i=1}^m} % mean from i=1 to n
\newcommand{\meankg}{\frac{1}{g} \sum\limits_{k=1}^g} % mean from k=1 to g
\newcommand{\prodin}{\prod\limits_{i=1}^n} % product from i=1 to n
\newcommand{\prodkg}{\prod\limits_{k=1}^g} % product from k=1 to g
\newcommand{\prodjp}{\prod\limits_{j=1}^p} % product from j=1 to p

% linear algebra
\newcommand{\one}{\bm{1}} % 1, unitvector
\newcommand{\zero}{\mathbf{0}} % 0-vector
\newcommand{\id}{\bm{I}} % I, identity
\newcommand{\diag}{\operatorname{diag}} % diag, diagonal
\newcommand{\trace}{\operatorname{tr}} % tr, trace
\newcommand{\spn}{\operatorname{span}} % span
\newcommand{\scp}[2]{\left\langle #1, #2 \right\rangle} % <.,.>, scalarproduct
\newcommand{\mat}[1]{\begin{pmatrix} #1 \end{pmatrix}} % short pmatrix command
\newcommand{\Amat}{\mathbf{A}} % matrix A
\newcommand{\Deltab}{\mathbf{\Delta}} % error term for vectors

% basic probability + stats
\renewcommand{\P}{\mathds{P}} % P, probability
\newcommand{\E}{\mathds{E}} % E, expectation
\newcommand{\var}{\mathsf{Var}} % Var, variance
\newcommand{\cov}{\mathsf{Cov}} % Cov, covariance
\newcommand{\corr}{\mathsf{Corr}} % Corr, correlation
\newcommand{\normal}{\mathcal{N}} % N of the normal distribution
\newcommand{\iid}{\overset{i.i.d}{\sim}} % dist with i.i.d superscript
\newcommand{\distas}[1]{\overset{#1}{\sim}} % ... is distributed as ...

% machine learning
\newcommand{\Xspace}{\mathcal{X}} % X, input space
\newcommand{\Yspace}{\mathcal{Y}} % Y, output space
\newcommand{\Zspace}{\mathcal{Z}} % Z, space of sampled datapoints
\newcommand{\nset}{\{1, \ldots, n\}} % set from 1 to n
\newcommand{\pset}{\{1, \ldots, p\}} % set from 1 to p
\newcommand{\gset}{\{1, \ldots, g\}} % set from 1 to g
\newcommand{\Pxy}{\mathbb{P}_{xy}} % P_xy
\newcommand{\Exy}{\mathbb{E}_{xy}} % E_xy: Expectation over random variables xy
\newcommand{\xv}{\mathbf{x}} % vector x (bold)
\newcommand{\xtil}{\tilde{\mathbf{x}}} % vector x-tilde (bold)
\newcommand{\yv}{\mathbf{y}} % vector y (bold)
\newcommand{\xy}{(\xv, y)} % observation (x, y)
\newcommand{\xvec}{\left(x_1, \ldots, x_p\right)^\top} % (x1, ..., xp)
\newcommand{\Xmat}{\mathbf{X}} % Design matrix
\newcommand{\allDatasets}{\mathds{D}} % The set of all datasets
\newcommand{\allDatasetsn}{\mathds{D}_n}  % The set of all datasets of size n
\newcommand{\D}{\mathcal{D}} % D, data
\newcommand{\Dn}{\D_n} % D_n, data of size n
\newcommand{\Dtrain}{\mathcal{D}_{\text{train}}} % D_train, training set
\newcommand{\Dtest}{\mathcal{D}_{\text{test}}} % D_test, test set
\newcommand{\xyi}[1][i]{\left(\xv^{(#1)}, y^{(#1)}\right)} % (x^i, y^i), i-th observation
\newcommand{\Dset}{\left( \xyi[1], \ldots, \xyi[n]\right)} % {(x1,y1)), ..., (xn,yn)}, data
\newcommand{\defAllDatasetsn}{(\Xspace \times \Yspace)^n} % Def. of the set of all datasets of size n
\newcommand{\defAllDatasets}{\bigcup_{n \in \N}(\Xspace \times \Yspace)^n} % Def. of the set of all datasets
\newcommand{\xdat}{\left\{ \xv^{(1)}, \ldots, \xv^{(n)}\right\}} % {x1, ..., xn}, input data
\newcommand{\ydat}{\left\{ \yv^{(1)}, \ldots, \yv^{(n)}\right\}} % {y1, ..., yn}, input data
\newcommand{\yvec}{\left(y^{(1)}, \hdots, y^{(n)}\right)^\top} % (y1, ..., yn), vector of outcomes
\newcommand{\greekxi}{\xi} % Greek letter xi
\renewcommand{\xi}[1][i]{\xv^{(#1)}} % x^i, i-th observed value of x
\newcommand{\yi}[1][i]{y^{(#1)}} % y^i, i-th observed value of y
\newcommand{\xivec}{\left(x^{(i)}_1, \ldots, x^{(i)}_p\right)^\top} % (x1^i, ..., xp^i), i-th observation vector
\newcommand{\xj}{\xv_j} % x_j, j-th feature
\newcommand{\xjvec}{\left(x^{(1)}_j, \ldots, x^{(n)}_j\right)^\top} % (x^1_j, ..., x^n_j), j-th feature vector
\newcommand{\phiv}{\mathbf{\phi}} % Basis transformation function phi
\newcommand{\phixi}{\mathbf{\phi}^{(i)}} % Basis transformation of xi: phi^i := phi(xi)

%%%%%% ml - models general
\newcommand{\lamv}{\bm{\lambda}} % lambda vector, hyperconfiguration vector
\newcommand{\Lam}{\bm{\Lambda}}	 % Lambda, space of all hpos
% Inducer / Inducing algorithm
\newcommand{\preimageInducer}{\left(\defAllDatasets\right)\times\Lam} % Set of all datasets times the hyperparameter space
\newcommand{\preimageInducerShort}{\allDatasets\times\Lam} % Set of all datasets times the hyperparameter space
% Inducer / Inducing algorithm
\newcommand{\ind}{\mathcal{I}} % Inducer, inducing algorithm, learning algorithm

% continuous prediction function f
\newcommand{\ftrue}{f_{\text{true}}}  % True underlying function (if a statistical model is assumed)
\newcommand{\ftruex}{\ftrue(\xv)} % True underlying function (if a statistical model is assumed)
\newcommand{\fx}{f(\xv)} % f(x), continuous prediction function
\newcommand{\fdomains}{f: \Xspace \rightarrow \R^g} % f with domain and co-domain
\newcommand{\Hspace}{\mathcal{H}} % hypothesis space where f is from
\newcommand{\fbayes}{f^{\ast}} % Bayes-optimal model
\newcommand{\fxbayes}{f^{\ast}(\xv)} % Bayes-optimal model
\newcommand{\fkx}[1][k]{f_{#1}(\xv)} % f_j(x), discriminant component function
\newcommand{\fh}{\hat{f}} % f hat, estimated prediction function
\newcommand{\fxh}{\fh(\xv)} % fhat(x)
\newcommand{\fxt}{f(\xv ~|~ \thetav)} % f(x | theta)
\newcommand{\fxi}{f\left(\xv^{(i)}\right)} % f(x^(i))
\newcommand{\fxih}{\hat{f}\left(\xv^{(i)}\right)} % f(x^(i))
\newcommand{\fxit}{f\left(\xv^{(i)} ~|~ \thetav\right)} % f(x^(i) | theta)
\newcommand{\fhD}{\fh_{\D}} % fhat_D, estimate of f based on D
\newcommand{\fhDtrain}{\fh_{\Dtrain}} % fhat_Dtrain, estimate of f based on D
\newcommand{\fhDnlam}{\fh_{\Dn, \lamv}} %model learned on Dn with hp lambda
\newcommand{\fhDlam}{\fh_{\D, \lamv}} %model learned on D with hp lambda
\newcommand{\fhDnlams}{\fh_{\Dn, \lamv^\ast}} %model learned on Dn with optimal hp lambda
\newcommand{\fhDlams}{\fh_{\D, \lamv^\ast}} %model learned on D with optimal hp lambda

% discrete prediction function h
\newcommand{\hx}{h(\xv)} % h(x), discrete prediction function
\newcommand{\hh}{\hat{h}} % h hat
\newcommand{\hxh}{\hat{h}(\xv)} % hhat(x)
\newcommand{\hxt}{h(\xv | \thetav)} % h(x | theta)
\newcommand{\hxi}{h\left(\xi\right)} % h(x^(i))
\newcommand{\hxit}{h\left(\xi ~|~ \thetav\right)} % h(x^(i) | theta)
\newcommand{\hbayes}{h^{\ast}} % Bayes-optimal classification model
\newcommand{\hxbayes}{h^{\ast}(\xv)} % Bayes-optimal classification model

% yhat
\newcommand{\yh}{\hat{y}} % yhat for prediction of target
\newcommand{\yih}{\hat{y}^{(i)}} % yhat^(i) for prediction of ith targiet
\newcommand{\resi}{\yi- \yih}

% theta
\newcommand{\thetah}{\hat{\theta}} % theta hat
\newcommand{\thetav}{\bm{\theta}} % theta vector
\newcommand{\thetavh}{\bm{\hat\theta}} % theta vector hat
\newcommand{\thetat}[1][t]{\thetav^{[#1]}} % theta^[t] in optimization
\newcommand{\thetatn}[1][t]{\thetav^{[#1 +1]}} % theta^[t+1] in optimization
\newcommand{\thetahDnlam}{\thetavh_{\Dn, \lamv}} %theta learned on Dn with hp lambda
\newcommand{\thetahDlam}{\thetavh_{\D, \lamv}} %theta learned on D with hp lambda
\newcommand{\mint}{\min_{\thetav \in \Theta}} % min problem theta
\newcommand{\argmint}{\argmin_{\thetav \in \Theta}} % argmin theta

% densities + probabilities
% pdf of x
\newcommand{\pdf}{p} % p
\newcommand{\pdfx}{p(\xv)} % p(x)
\newcommand{\pixt}{\pi(\xv~|~ \thetav)} % pi(x|theta), pdf of x given theta
\newcommand{\pixit}[1][i]{\pi\left(\xi[#1] ~|~ \thetav\right)} % pi(x^i|theta), pdf of x given theta
\newcommand{\pixii}[1][i]{\pi\left(\xi[#1]\right)} % pi(x^i), pdf of i-th x

% pdf of (x, y)
\newcommand{\pdfxy}{p(\xv,y)} % p(x, y)
\newcommand{\pdfxyt}{p(\xv, y ~|~ \thetav)} % p(x, y | theta)
\newcommand{\pdfxyit}{p\left(\xi, \yi ~|~ \thetav\right)} % p(x^(i), y^(i) | theta)

% pdf of x given y
\newcommand{\pdfxyk}[1][k]{p(\xv | y= #1)} % p(x | y = k)
\newcommand{\lpdfxyk}[1][k]{\log p(\xv | y= #1)} % log p(x | y = k)
\newcommand{\pdfxiyk}[1][k]{p\left(\xi | y= #1 \right)} % p(x^i | y = k)

% prior probabilities
\newcommand{\pik}[1][k]{\pi_{#1}} % pi_k, prior
\newcommand{\lpik}[1][k]{\log \pi_{#1}} % log pi_k, log of the prior
\newcommand{\pit}{\pi(\thetav)} % Prior probability of parameter theta

% posterior probabilities
\newcommand{\post}{\P(y = 1 ~|~ \xv)} % P(y = 1 | x), post. prob for y=1
\newcommand{\postk}[1][k]{\P(y = #1 ~|~ \xv)} % P(y = k | y), post. prob for y=k
\newcommand{\pidomains}{\pi: \Xspace \rightarrow \unitint} % pi with domain and co-domain
\newcommand{\pibayes}{\pi^{\ast}} % Bayes-optimal classification model
\newcommand{\pixbayes}{\pi^{\ast}(\xv)} % Bayes-optimal classification model
\newcommand{\pix}{\pi(\xv)} % pi(x), P(y = 1 | x)
\newcommand{\piv}{\bm{\pi}} % pi, bold, as vector
\newcommand{\pikx}[1][k]{\pi_{#1}(\xv)} % pi_k(x), P(y = k | x)
\newcommand{\pikxt}[1][k]{\pi_{#1}(\xv ~|~ \thetav)} % pi_k(x | theta), P(y = k | x, theta)
\newcommand{\pixh}{\hat \pi(\xv)} % pi(x) hat, P(y = 1 | x) hat
\newcommand{\pikxh}[1][k]{\hat \pi_{#1}(\xv)} % pi_k(x) hat, P(y = k | x) hat
\newcommand{\pixih}{\hat \pi(\xi)} % pi(x^(i)) with hat
\newcommand{\pikxih}[1][k]{\hat \pi_{#1}(\xi)} % pi_k(x^(i)) with hat
\newcommand{\pdfygxt}{p(y ~|~\xv, \thetav)} % p(y | x, theta)
\newcommand{\pdfyigxit}{p\left(\yi ~|~\xi, \thetav\right)} % p(y^i |x^i, theta)
\newcommand{\lpdfygxt}{\log \pdfygxt } % log p(y | x, theta)
\newcommand{\lpdfyigxit}{\log \pdfyigxit} % log p(y^i |x^i, theta)

% probababilistic
\newcommand{\bayesrulek}[1][k]{\frac{\P(\xv | y= #1) \P(y= #1)}{\P(\xv)}} % Bayes rule
\newcommand{\muk}{\bm{\mu_k}} % mean vector of class-k Gaussian (discr analysis)

% residual and margin
\newcommand{\eps}{\epsilon} % residual, stochastic
\newcommand{\epsv}{\bm{\epsilon}} % residual, stochastic, as vector
\newcommand{\epsi}{\epsilon^{(i)}} % epsilon^i, residual, stochastic
\newcommand{\epsh}{\hat{\epsilon}} % residual, estimated
\newcommand{\epsvh}{\hat{\epsv}} % residual, estimated, vector
\newcommand{\yf}{y \fx} % y f(x), margin
\newcommand{\yfi}{\yi \fxi} % y^i f(x^i), margin
\newcommand{\Sigmah}{\hat \Sigma} % estimated covariance matrix
\newcommand{\Sigmahj}{\hat \Sigma_j} % estimated covariance matrix for the j-th class

% ml - loss, risk, likelihood
\newcommand{\Lyf}{L\left(y, f\right)} % L(y, f), loss function
\newcommand{\Lypi}{L\left(y, \pi\right)} % L(y, pi), loss function
\newcommand{\Lxy}{L\left(y, \fx\right)} % L(y, f(x)), loss function
\newcommand{\Lxyi}{L\left(\yi, \fxi\right)} % loss of observation
\newcommand{\Lxyt}{L\left(y, \fxt\right)} % loss with f parameterized
\newcommand{\Lxyit}{L\left(\yi, \fxit\right)} % loss of observation with f parameterized
\newcommand{\Lxym}{L\left(\yi, f\left(\bm{\tilde{x}}^{(i)} ~|~ \thetav\right)\right)} % loss of observation with f parameterized
\newcommand{\Lpixy}{L\left(y, \pix\right)} % loss in classification
\newcommand{\Lpiv}{L\left(y, \piv\right)} % loss in classification
\newcommand{\Lpixyi}{L\left(\yi, \pixii\right)} % loss of observation in classification
\newcommand{\Lpixyt}{L\left(y, \pixt\right)} % loss with pi parameterized
\newcommand{\Lpixyit}{L\left(\yi, \pixit\right)} % loss of observation with pi parameterized
\newcommand{\Lhxy}{L\left(y, \hx\right)} % L(y, h(x)), loss function on discrete classes
\newcommand{\Lr}{L\left(r\right)} % L(r), loss defined on residual (reg) / margin (classif)
\newcommand{\lone}{|y - \fx|} % L1 loss
\newcommand{\ltwo}{\left(y - \fx\right)^2} % L2 loss
\newcommand{\lbernoullimp}{\ln(1 + \exp(-y \cdot \fx))} % Bernoulli loss for -1, +1 encoding
\newcommand{\lbernoullizo}{- y \cdot \fx + \log(1 + \exp(\fx))} % Bernoulli loss for 0, 1 encoding
\newcommand{\lcrossent}{- y \log \left(\pix\right) - (1 - y) \log \left(1 - \pix\right)} % cross-entropy loss
\newcommand{\lbrier}{\left(\pix - y \right)^2} % Brier score
\newcommand{\risk}{\mathcal{R}} % R, risk
\newcommand{\riskbayes}{\mathcal{R}^\ast}
\newcommand{\riskf}{\risk(f)} % R(f), risk
\newcommand{\riskdef}{\E_{y|\xv}\left(\Lxy \right)} % risk def (expected loss)
\newcommand{\riskt}{\mathcal{R}(\thetav)} % R(theta), risk
\newcommand{\riske}{\mathcal{R}_{\text{emp}}} % R_emp, empirical risk w/o factor 1 / n
\newcommand{\riskeb}{\bar{\mathcal{R}}_{\text{emp}}} % R_emp, empirical risk w/ factor 1 / n
\newcommand{\riskef}{\riske(f)} % R_emp(f)
\newcommand{\risket}{\mathcal{R}_{\text{emp}}(\thetav)} % R_emp(theta)
\newcommand{\riskr}{\mathcal{R}_{\text{reg}}} % R_reg, regularized risk
\newcommand{\riskrt}{\mathcal{R}_{\text{reg}}(\thetav)} % R_reg(theta)
\newcommand{\riskrf}{\riskr(f)} % R_reg(f)
\newcommand{\riskrth}{\hat{\mathcal{R}}_{\text{reg}}(\thetav)} % hat R_reg(theta)
\newcommand{\risketh}{\hat{\mathcal{R}}_{\text{emp}}(\thetav)} % hat R_emp(theta)
\newcommand{\LL}{\mathcal{L}} % L, likelihood
\newcommand{\LLt}{\mathcal{L}(\thetav)} % L(theta), likelihood
\newcommand{\LLtx}{\mathcal{L}(\thetav | \xv)} % L(theta|x), likelihood
\newcommand{\logl}{\ell} % l, log-likelihood
\newcommand{\loglt}{\logl(\thetav)} % l(theta), log-likelihood
\newcommand{\logltx}{\logl(\thetav | \xv)} % l(theta|x), log-likelihood
\newcommand{\errtrain}{\text{err}_{\text{train}}} % training error
\newcommand{\errtest}{\text{err}_{\text{test}}} % test error
\newcommand{\errexp}{\overline{\text{err}_{\text{test}}}} % avg training error

% lm
\newcommand{\thx}{\thetav^\top \xv} % linear model
\newcommand{\olsest}{(\Xmat^\top \Xmat)^{-1} \Xmat^\top \yv} % OLS estimator in LM

%%% HPO Basics 
\newcommand{\Ilam}{\ind_{\lamv}} % inducer with HP
\newcommand{\LamS}{\tilde\Lam} % search space
\newcommand{\lami}[1][i]{\lamv^{(#1)}} % lambda i
\newcommand{\clam}{c(\lamv)} % c(lambda)
\newcommand{\clamh}{c(\lamh)} % c(lambda-hat)
\newcommand{\lams}{\lamv^{*}} % theoretical min of c
\newcommand{\lamh}{\hat{\lamv}} % returned lambda of HPO
\newcommand{\lamp}{\lamv^+} % proposed lambda
\newcommand{\clamp}{c(\lamp)} % c of proposed lambda
\newcommand{\archive}{\mathcal{A}} % archive
\newcommand{\archivet}[1][t]{\mathcal{A}^{[#1]}} % archive at time step t

\newcommand{\tuner}{\mathcal{T}} % tuner
\newcommand{\tunerfull}{\tuner_{\ind,\LamS, \rho,\JJ}} % tuner with inducer, search space, perf measure, resampling strategy

%%% Bayesian Opt
\newcommand{\chlam}{\hat{c}(\lamv)} % post mean of SM 
\newcommand{\shlam}{\hat{\sigma}(\lamv)} % post sd of SM
\newcommand{\vhlam}{\hat{\sigma}^2(\lamv)} % post var of SM
\newcommand{\ulam}{u(\lamv)} % acquisition function
\newcommand{\lambdaopt}{\lambda^{*}} % minimum of the black box function Psi
\newcommand{\metadata}{\left\{\left(\lami, \Psi^{[i]}\right)\right\}} % metadata for the Gaussian process
\newcommand{\lamvec}{\left(\lambda^{[1]}, \dots, \lambda^{[\minit]}\right)} % vector of different inputs
\newcommand{\minit}{m_{\text{init}}} % size of the initial design

%%% Multifidelity / Hyperband
\newcommand{\lambu}{\lambda_{\text{budget}}} % single lambda_budget component HP
\newcommand{\lamfid}{\lambda_{\text{fid}}} % single lambda fidelity
\newcommand{\lamfidl}{\lamfid^{\textrm{low}}} % single lambda fidelity lower
\newcommand{\lamfidu}{\lamfid^{\textrm{upp}}} % single lambda fidelity upper
\newcommand{\etahb}{\eta_{\text{HB}}}  % HB multiplier eta 


\newcommand{\titlefigure}{figure_man/math_robot.jpg}
\newcommand{\learninggoals}{
	\item Refresher on the basics of probability theory
%	\item Bayes risk
%    \item Consistent learners
%    \item Bayes regret, estimation and approximation error    
%	\item Optimal constant model
%	\item Proper scoring rules
}

\title{Supervised Learning}
% \author{Bernd Bischl, Christoph Molnar, Daniel Schalk, Fabian Scheipl}
\institute{\href{https://compstat-lmu.github.io/lecture_i2ml/}{compstat-lmu.github.io/lecture\_i2ml}}
\date{}


\begin{document}

\lecturechapter{Refreshing Mathematical Tools}
\lecture{Supervised Learning}

\newcommand{\F}{\mathcal{F}}

\begin{vbframe}{Probability Space}
	
	\textbf{Probability space.} A probability space is a triple $(\Omega,\F,\P)$ which is modeling a specific random experiment/process.
		%		
		The components are
		%		
		\begin{itemize}
			%			
			\item a sample space $\Omega,$ which is the set of all possible outcomes of the random process modeled by the probability space.
			%			
			\item a $\sigma$-algebra $\F,$ which is a family of sets representing the allowable events of the random process modeled by the probability space.
			In particular, each set in $\F$ is a subset of the sample space $\Omega.$
			%
			\item a probability measure $\P,$ which assigns each allowable event a probability value, i.e., $\P:\F\to[0,1].$ It satisfies the following axioms of probability:
			%			
			\begin{itemize}
				%				
				\item  \emph{Completeness ---} $\P(\Omega)=1,$  
				%				
				\item  \emph{$\sigma$-additivity ---} For any finite or countably infinite sequence of mutually disjoint events $E_1,E_2,\ldots,$ it holds that
				%				
				$$\P(\cup_{i\geq1}  E_i) = \sum\nolimits_{i\geq 1} \P(E_i).$$
				%
			\end{itemize}
			%
		\end{itemize}
		%		
		\framebreak
		Examples:
		%		
		\begin{itemize}
			%			
			\item \emph{Coin Tossing ---} Possible outcomes are $\Omega=\{\texttt{H},\texttt{T} \}$ with \texttt{H} resp.\ \texttt{T} representing ''heads'' resp.\ ''tails''. The allowable events are contained in $\F=\{\varnothing,  \{\texttt{H}\} , \{\texttt{T}\}, \{\texttt{H},\texttt{T}\}  \}.$ If the coin is fair, then $\P(\texttt{H})=\P(\texttt{T})=1/2.$
			%			
			\item \emph{Dice Rolling---} Possible outcomes are $\Omega=\{1,2,3,4,5,6\}.$  The allowable events are contained in $2^\Omega,$ i.e., the power set of $\Omega.$
			If the dice is fair, then $\P(i)=1/6$ for $i=1,\ldots,6.$
			%			
		\end{itemize}
		%
		\textbf{Further properties.} For any probability space $(\Omega,\F,\P)$ the following properties hold
		%		
		\begin{itemize}
			%			
			\item \emph{Monotonicity ---} $A,B\in \F,$ and $A\subset B,$ then $\P(A)\leq \P(B).$
			%			
			\item \emph{Union bound ---}  For any finite or countably infinite sequence events $E_1,E_2,\ldots,$ 
			%			
			$$\P(\cup_{i\geq1}  E_i) \leq \sum\nolimits_{i\geq 1} \P(E_i).$$
			%			
		\end{itemize}

\end{vbframe}


\begin{vbframe}{Running Exercise}  
An urn contains five blue, three green, and one red ball. Two balls are randomly selected (without replacement).

\lz



What is the sample space of this experiment?


\lz

\lz
\lz
\lz

What is the probability of each point in the sample space?



\end{vbframe}

\begin{vbframe}{Independence}  


\textbf{Independence of events:} 
%		
\begin{itemize}
	%		
	\item 		Two events $A,B\in \F$ are \emph{independent} iff $\P(A \cap B)=\P(A)\P(B).$
	%			
	\item Events $E_1,\ldots,E_n \in \F$ are called \emph{pairwise independent} iff any pair $E_i,E_j$ with $i \neq j$ is independent.
	%	
	\item Events $E_1,\ldots,E_n \in \F$ are called \emph{mutually independent} iff for any  subset $I \subset \{1,\ldots,n\}$ it holds that  $\P(\bigcap_{i \in I} E_i )= \prod_{i \in I}\P(E_i).$
	%		
\end{itemize}
%		
Note that pairwise independence does not imply mutual independence! 

\lz

Example: One urn with four balls with the labels $110,101,011,000.$ We select one ball randomly.
\begin{itemize}
%	
	\item For $i=1,2,3$ let $E_i:=\{ \mbox{Selected ball has zero at the $i$-th position}  \}.$ 
%	
	\item $\P(E_i) = 1/2,$ $i=1,2,3$ and $\P(E_1 \cap E_2) = \P(E_1 \cap E_3) = \P(E_2 \cap E_3) =1/4.$
%	
	\item But $\P(E_1 \cap E_2 \cap E_3) = 1/4 \neq 1/8 = \P(E_1)\P(E_2)\P(E_3).$
%	
\end{itemize}
 


\framebreak

%
\textbf{Conditional probability.} The \emph{conditional probability} that event $A\in \F$ occurs given that event $B\in\F$ occurs is $\P(A|B)= \frac{\P(A\cap B)}{\P(B)}.$ The conditional probability is well-defined only if $\P(B) > 0.$


%	
\textbf{Bayes rule.} For two events $A,B\in \F$  it holds that $$\P(A|B)=\frac{\P(B|A) \P(A)}{\P(B)}.$$
%
\lz 

%
\textbf{The law of total probability.} Let $E_1,\ldots,E_n \in \F$  be mutually disjoint events, such that $\cup_{i=1}^n E_i = \Omega,$ then $ \forall A\in \F,$ {  $$\P(A) = \sum_{i=1}^n \P(A \cap E_i) = \sum_{i=1}^n \P(A | E_i) \P(E_i).$$}
%	

\end{vbframe}


\begin{vbframe}{Running Exercise}  
	An urn contains five blue, three green, and one red ball. Two balls are randomly selected (without replacement).
	
	\lz
	
	
	
	Consider the event $A=\{\mbox{First ball is red}\}$ and the event $B=\{\mbox{Second ball is red}\}.$ 
	Are these events independent?
	
	
	\lz
	
	\lz
	\lz
	\lz
	
	Are the events independent if we put a ball back into the urn after each selection?
	
	
	
\end{vbframe}


\begin{vbframe}{Random variables}  
 
 \textbf{Random variables.} A random variable $X$ on a sample space $\Omega$ is a real-valued function on $\Omega,$ that is $X:\Omega \to \R.$ The following observations can be made:
 %
 \begin{itemize}
 	%		
 	\item 	A random variable defines a probability space $(\Omega_X,\F_X,\P_X)$ with $\Omega_X= Im(X)$ and $\P_X(A)=\P(\{\omega\in\Omega \, | \, X(\omega) \in A \})$ for any $A\subset \Omega_X.$ Usually, one writes just $\P(X\in A)$ to denote the latter term, which is the \emph{probability distribution of $X$.}\\
 	
 	{\tiny (Technical remark: $\F_X$ is usually the Borel-$\sigma$-algebra on $\R$.)}
 	%	
 	\item In practical applications oftentimes the original probability space $(\Omega,\F,\P)$ is not the interesting object, but rather the induced probability space by $X.$ 
 	%		
 	One is rather interested in the probability of the outcome of the random variable:
 	%		
 	\begin{itemize}
 		%			
 		\item \emph{Coin tossing ---} Let $X$ be the random variable counting the number of tails after 100 flips.
 		%			
 		%			The 
 		%			
 		\item \emph{Financial market ---} Let $X_t$ be the  price of some asset in a future time $t.$
 		%			
 		%			
 	\end{itemize}
 	%	
 	\framebreak
 	%
 	\item Functions of random variables are again random variables, i.e., if $f:\R \to \R$ is some (measurable) function, then $Z=f(X)$ is also a random variable.\\
 	%	
% 	{\tiny (Technical remark: to be rigorously one has to require that the function is measurable.)}
 	%	
 	\item Identically distributed --- Two random variables $X$ and $Y$ are identically distributed if their probability distributions coincide, i.e., $\P_X = \P_Y.$ 
 	%		
 	%	\item The \emph{cumulative distribution function} of a random variable  determines the probability that the random variable will take a value smaller than or equal a specific value.
 	%%	\\
 	%	More specifically, $F_X:\R \to [0,1]$ with $F_X(x) := \P(X \leq x)$ for any $x\in \R$ is the cumulative distribution function of the random variable $X.$
 	%
 \end{itemize}
 %	
 One distinguishes between two types of random variables:
 %	
 \begin{itemize}
 	%		
 	
 	\item A \emph{discrete random variable} is a random variable that can take only a finite or countably
 	infinite number of values. Its probability distribution is determined by the \emph{probability mass function}  which assigns a probability to each value in the image of $X.$
 	%	
 	\item A \emph{continuous random variable} is a random variable which can take uncountably
 	infinite number of values. Usually its probability distribution is determined by a \emph{density function,} which assigns probabilities to intervals of the image of $X.$
 	%		
 \end{itemize}


\end{vbframe}



\begin{vbframe}{Discrete random variables} 

If the image $\Omega_X$ of $X$ is discrete (e.g., finite or countably infinite), then $X$ is called a \emph{discrete RV}.
 
 \lz
 
 For a discrete RV $X$, the function 
 $$p:\, \Omega_X \to [0,1], \,  x \mapsto  \P \left(X \in \{x\} \right) = \P \left(X = x \right)
 $$ 
 is called a probability function or probability mass function of $X$. 
 
 
 \lz 
 
 Obviously, $p(x) \geq 0$ and $\sum_{x \in \Omega_X} p(x) = 1$.
% 
 
	Examples:
	\begin{itemize}
		\item \emph{Bernoulli distribution}: For a binary RV with $\Omega_X = \{0,1\}$, $X \sim \mathrm{Ber}(\theta)$ if $p(1) = \theta$ and $p(0)= 1 - \theta$.
		\item \emph{Binomial distribution}: $X \sim \mathrm{Bin}(n,\theta)$ if $$
p(k) = \left\{ \begin{array}{cl}
	{n \choose k} \theta^k (1-\theta)^{n-k} & \text{ if } k \in \{0, \ldots , n\}\\
	0 & \text{ otherwise}
\end{array} \right. \, .
$$
	\end{itemize}
 


\end{vbframe}


\begin{vbframe}{Continuous random variables} 
	
	$X$ is a \emph{continuous RV} if $\Omega_X$ is non-discrete and there exists a function $p: \R \to \R$ such that 
%	
	\begin{itemize}
%		
		\item 
		$p(x) \geq 0$ for all $x \in \mathbb{R}$
%		
		\item $\int_{-\infty}^\infty p(x) \, dx = 1$,
%		
		\item for all $a \leq b$ it holds that	$
		\P(a \leq X \leq b) = \int_a^b p(x) \, dx .
		$
%
	\end{itemize} 
% 
	The function $p$ is called the probability density function (PDF) of $X$.
	
	\lz
	
	Examples.
	
	\begin{itemize}
			\item \emph{Uniform distribution}: $X \sim U(a,b)$ if 
		$$
		p(x) = \left\{ \begin{array}{cl}
			1/(b-a) & \text{ if } a \leq x \leq b  \\
			0 & \text{ otherwise}
		\end{array} \right. .
		$$
		\item \emph{Normal/Gaussian distribution}: $X \sim \normal(\mu, \sigma)$ if
		$$
		p(x) = \frac{1}{\sigma \sqrt{2\pi}} \exp \left( -\frac{(x- \mu)^2 }{2 \sigma^2} \right)\, .
		$$ 
	\end{itemize}
\end{vbframe}

\begin{vbframe}{Cumulative distribution function}
 
 The cumulative distribution function (CDF) of a random variable $X$ is the function
		$$
		F_X:\, \mathbb{R} \to [0,1], \, x \mapsto \P(X \leq x) \, .
		$$
		
A CDF fully characterizes a RV: If $F_X(x) = F_Y(x)$ for all $x \in \mathbb{R}$, then $X$ and $Y$ are identically distributed.
		
		\lz
		
%		A function $F:\, \mathbb{R} \to [0,1]$ is a CDF for some probability measure $\P$ if and only if the following properties are satisfied: 
%\begin{itemize}
%	\item $F$ is non-decreasing, i.e., $x_1 < x_2$ implies $F(x_1) \leq F(x_2)$;
%	\item $F$ is normalized: $\lim_{x \rightarrow - \infty} F(x) = 0$ and $\lim_{x \rightarrow \infty} F(x) = 1$;
%	\item $F$ is right-continuous, i.e., $\lim_{y \downarrow x} F(y) = F(x)$ for all $x \in \mathbb{R}$.
%\end{itemize}

If $X$ is 
\begin{itemize}
%	
	\item discrete with probability mass function $p$, then for all $x \in \mathbb{R}$, 
	$
	F_X(x) = \sum_{y \in \Omega_X \cap (-\infty, x]} p(y) \, .
	$
%	
	\item continuous with probability density function $p$, then 
	$
	F_X(x)  = \int_{-\infty}^x p(t) \, dt 
	$
	for all $x \in \mathbb{R}$, and $p(x) = F_X'(x)$ whenever $F_X$ is differentiable at $x$. 
%	
\end{itemize}

\end{vbframe}


\begin{vbframe}{Running Exercise}  
	An urn contains five blue, three green, and one red ball. Two balls are randomly selected (without replacement).
	
	\lz
	
	
	
	Let $X$ be the number of green balls selected. What are the possible values of $X?$
	
	
	\lz
	
	\lz
	\lz
	\lz
	
	What is the cumulative distribution function of $X$?
	
	
	
\end{vbframe}

\begin{frame}[t]{Expected value/Expectation} 
%	
	Expectation is the most basic characteristic of a random variable. Let $X$ be a random variable, then the expectation of $X$, denoted by $\E(X),$ is 
	%	
	$$	\E(X) = \int x \, dF(x) = \left\{ \begin{array}{ll}
		\sum_{x \in \Omega_X} x \, p(x) & \text{ if } X \text{ is discrete} \\
		\int_{\Omega_X} x \, p(x) \, dx& \text{ if } X \text{ is continuous} 
	\end{array} \right.  	$$
	%	
 provided the sum resp.\ the integral is well-defined and exists.
	
	%	
	Some important properties of the expected value are:
	%
	\begin{itemize}
		%		
		\item Linearity --- For any constants $c_1,c_2 \in \R$ and any pair of random variables $X$ and $Y$ it holds that $ \E(c_1 X + c_2 Y) = c_1 \E(X) + c_2 \E(Y).$
		%		 
%		\item Independence --- If $X$ and $Y$ are independent random variables, then $\E(XY)=\E(X)\E(Y).$
		%		
		\item Transformations --- If $f:\R \to \R$  is a (measurable) function, then the expectation of $f(X)$ is 
		%		
		$$\E(f(X)) = \int f(x) \, dF(x) = \left\{ \begin{array}{ll}
			\sum_{x \in \Omega_X} f(x) \, p(x) & \text{ if } X \text{ is discrete} \\
			\int_{\Omega_X} f(x) \, p(x) \, dx& \text{ if } X \text{ is continuous} 
		\end{array} \right.  	$$   
		%		 
		{\tiny (provided the sum resp.\ integral exists.)}
		% 
		%		
	\end{itemize}
\end{frame}


\begin{vbframe}{Running Exercise}  
	An urn contains five blue, three green, and one red ball. Two balls are randomly selected (without replacement).
	
	\lz
	
	
	
	Let $X$ be the number of green balls selected. What is the expected value of $X?$
	
	 
	
	
	
\end{vbframe}


\begin{vbframe}{Variance and covariance}
%	 
		The variance of a RV $X$ is defined as follows:
		$$
		\var(X) = \E \left[ (X - \E(X))^2 \right] = \int_{\Omega_X} (x-\E(X))^2  \, dF(x)  \, ,
		$$
		provided the integral on the right-hand side exists. 
		
		\lz
		 
		The \emph{standard deviation} is defined by $\sqrt{\var(X)}$.
		
		
		\lz
		 
		The \emph{covariance} between RVs $X$ and $Y$ is  
		$$
		\cov(X,Y) = \E \left[  (X- \E[X]) (Y - \E[Y]) \right]
		$$
%		 
\end{vbframe}



\begin{frame}[t]{Multivariate random variables}
	
	
	RVs $X_1  , \ldots  , X_n$ over the same probability space can be combined into a \emph{random vector} $\bm{X}=(X_1, \ldots , X_n)$. 
	
	\lz
	
	Their joint distribution is specified by the joint mass/density function $p_{\bm{X}}$, such that for any measurable set $A$ it holds that
	$$ \P(\bm{X}  \in A) =  \left\{ \begin{array}{ll}
		\sum_{  (x_1 , \ldots , x_n) \in A  } p_{\bm{X}}(x_1 , \ldots , x_n) & \text{ if } X_1,\ldots,X_n \text{ are discrete} \\
		\int_{  A}   p_{\bm{X}}(x_1 , \ldots , x_n) \, dx_1\ldots dx_n& \text{ if }  X_1,\ldots,X_n  \text{ are continuous} 
	\end{array} \right.      $$ 
%
	The marginal distribution $p_1$ of $X_1$ is given by
	\[
	p_1(x_1) =  \left\{ \begin{array}{ll}
		\sum_{  (x_2 , \ldots , x_n) \in \Omega_{X_2} \times \ldots \times \Omega_{X_n}  } p_{\bm{X}}(x_1 , \ldots , x_n) & \text{ if } \text{  discrete} \\
		\int_{  \Omega_{X_2} \times \ldots \times \Omega_{X_n} }   p_{\bm{X}}(x_1 , \ldots , x_n) \, dx_2\ldots dx_n& \text{ if }   \text{  continuous} 
	\end{array} \right.     
	\]
%	
	In the same way, the marginal distributions of $X_2, \ldots , X_n$ are defined. 
%	
 
	
%	\item 
	The same type of projection (summation/integration over all remaining variables) is used to define marginal distributions on subsets of variables $(X_{i_1} , \ldots  , X_{i_k})$ with 
	$\{ i_1 ,  \ldots  , i_k \} \subseteq \{1 , \ldots ,  n\}$.
	
	
\end{frame}



\begin{vbframe}{Running Exercise}  
	An urn contains five blue, three green, and one red ball. Two balls are randomly selected (without replacement).
	
	\lz
	
	
	
	Let $X_1$ be the number of green balls selected and $X_2$ the number of blue balls selected. What is the joint mass function of $(X_1,X_2)?$
	
	
	
	
	
\end{vbframe}


\begin{vbframe}{Independence of random variables}  
 
\begin{itemize}
	%		
	\item 	Two discrete random variables X and Y are independent iff for any $x\in \Omega_X,y\in \Omega_Y$ it holds that $	p_{XY}(x,y) =	p_X(x) p_Y(y).$
	%	
	\item Random variables $X_1, \ldots, X_n$ are pairwise independent iff for any pair $i,j$ and any $x_i\in \Omega_{X_i},x_j\in \Omega_{X_j}$ it holds that $	p_{X_iX_j}(x_i,x_j) =	p_{X_i}(x_i) p_{X_j}(x_j).$
%	$	\P(X_i=x_i, X_j=x_j) =	\P(X_i=x_i)\P(X_j=x_j).$
	%		
	\item Random variables  $X_1, \ldots, X_n$ are mutually independent iff for any subset $I \subset \{1,\ldots,n\}$ it holds that the joint probability mass/density function of $(X_i)_{i\in I}$ is given by
	$  \prod_{i \in I}	p_{X_i}(x_i)$  for any $x_i \in \Omega_{X_i},$ $i\in I.$
%	$	\P( \bigcap_{i \in I} X_i=x_i ) =	\prod_{i \in I}\P(X_i=x_i)$ for $x_i \in Im(X_i).$
	%		
\end{itemize}
%
Similar to independence of events, pairwise independence of random variables does not imply their
mutual independence. If we say that random variables are independent without further specifications we
are referring to mutual independence.
%
\framebreak

%
 Some important properties and concepts with respect to independence are:
%	
\begin{itemize}
	%		
	\item The iid assumption --- Random variables  $X_1, \ldots, X_n$ are called  \emph{independent and identically distributed} (iid) iff they are mutually independent and each random variable has the same probability distribution as the others.
	%		
	\item Independence under transformations --- Let $X$ and $Y$ be independent random variables and $f,g:\R \to \R$ are (measurable) functions. Then, $f(X)$ and $g(Y)$ are independent as well. 
	%		
\end{itemize}
%
\end{vbframe}



\begin{frame}[t]{Conditional distributions}
	 
		 If $(X,Y)$ have a joint distribution with mass function $p_{X,Y}$, then the \emph{conditional probability mass function} for $X$ given $Y$ is defined by
		$$
		p_{X|Y}(x ~|~ y) = \frac{p(X=x, Y=y)}{p(Y = y)}  = \frac{p_{X,Y}(x,y)}{p_Y(y)} 
		$$
		provided $p(Y = y) > 0$.
		\lz
		
		Likewise, in the continuous case, the \emph{conditional probability density function} is given by
		$$
		p_{X|Y}(x ~|~ y) = \frac{p_{X,Y}(x,y)}{p_Y(y)} 
		$$
		provided $p_Y(y) > 0$. Then,
		$$
		p(X \in A ~|~ Y= y) = \int_A  p_{X|Y}(x ~|~ y) \, dx \, .
		$$
		The soundness of this definition is less obvious than in the discrete case, due to conditioning on an event of probability 0 here. 
		
\end{frame}


\begin{vbframe}{Conditional expected value/expectation} 
%	
	Let $X$ and $Y$ be random variables, then the conditional expectation of $X$ given $Y=y$, denoted by $\E(X|Y=y),$ is given by
	%	
	$$	\E(X|Y=y) = \left\{ \begin{array}{ll}
		\sum_{x \in \Omega_X} x \, p_{X|Y}(x ~|~ y) & \text{ discrete case} \\
		\int_{\Omega_X} x \, p_{X|Y}(x ~|~ y) \, dx& \text{ continuous case} 
	\end{array} \right.  	$$
	%	
	
	%	
	\emph{Interpretation:} The expected value of $X$ under the condition that $Y=y$ holds.
	%	
	\lz
	
	Note that $\E(X|Y=y)$ induces a mapping from $\Omega_Y$ to $\R$ according to $y \mapsto \E(X|Y=y).$
	%	
	This function is called the \emph{conditional expectation of $X$ given $Y$} and simply denoted by $\E(X|Y).$
	%	
	Note that $\E(X|Y)$ is a random variable!
	%
	
	\lz
	
	$\E(X|Y)$ can be also interpreted as a prediction of $X$ under the information encoded by the random variable $Y.$
	
	\framebreak
	
	
	%
	Some important properties of the conditional expected value are the following. For any random variables $X,Y,Z$ it holds that
	%
	\begin{itemize}
		%		
		\item Linearity ---  For any constants $c_1,c_2 \in \R$  it holds that $ \E(c_1 X + c_2 Y | Z) = c_1 \E(X|Z) + c_2 \E(Y|Z).$
		%		
		\item Independence --- If $X$ and $Y$ are independent random variables, then $\E(X|Y)=\E(X).$
		%		
		%		
		\item Transformations --- If $f:\R \to \R$  is a (measurable) function, then the conditional expectation of $f(X)$ given $Y=y$ is 
		%		
		$$\E(f(X)|Y=y)  = \left\{ \begin{array}{ll}
			\sum_{x \in \Omega_X} f(x) \, p_{X|Y}(x ~|~ y) & \text{ discrete case} \\
			\int_{\Omega_X} f(x) \, p_{X|Y}(x ~|~ y) \, dx & \text{ continuous case} 
		\end{array} \right.  	$$
		%		 
		%
		\item Law of total expectation ---
		$\E(\E(X|Y)) = \E(X)$.  
%		{\tiny (Note: Here, the outer expectation on the left hand-side is taken with respect to the distribution of $Y$.)}
		%		 
		\item Tower property --- $ \E(\E(X|Y,Z)|Y) = \E(X|Y).$
		%		
	\end{itemize}
	
\end{vbframe}


\begin{frame}[t]{Conditional variance} 
	%	
	Let $X$ and $Y$ be random variables, then the conditional variance of $X$ given $Y=y$, denoted by $\var(X|Y=y),$ is given by
	%	
	$$	\var(X|Y=y)  = \E \left[ (X - \E[X~|~Y=y])^2  ~|~ Y=y  \right]	$$
	%	
	
	%	
	\emph{Interpretation:} Variance of the prediction $\E[X~|~Y]$ for $X.$
	%	
	\lz
	
	Note that $\var(X|Y=y)$ induces a mapping from $\Omega_Y$ to $\R$ according to $y \mapsto \var(X|Y=y).$
	%	
	This function is called the \emph{conditional variance of $X$ given $Y$} and simply denoted by $\var(X|Y).$
	%	
	Note that $\var(X|Y)$ is a random variable!
	%
	
	\lz
	
	An important property is the law of total variance:
	
	$$  \var(X) = \E(\var(X~|~Y)) + \var(\E(X~|~Y)).  $$

	
\end{frame}




\endlecture

\end{document} 
